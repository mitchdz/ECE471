\documentclass[12pt]{article}
\usepackage[margin=1in]{geometry}                % See geometry.pdf to learn the layout options. There are lots.
\geometry{letterpaper}                   % ... or a4paper or a5paper or ... 
%\geometry{landscape}                % Activate for for rotated page geometry
\usepackage[parfill]{parskip}    % Activate to begin paragraphs with an empty line rather than an indent

%%%%%%%%%%%%%%%%%%%%
\newcommand{\hide}[1]{}



\usepackage{natbib}
\usepackage{xcolor}
\usepackage{url}
\usepackage{hyperref}
\usepackage{mathtools}
\usepackage[utf8]{inputenc}
\usepackage{float}


\hide{
\usepackage{amscd}
\usepackage{amsfonts}
\usepackage{amsmath}
\usepackage{amssymb}
\usepackage{amsthm}
\usepackage{cases}		 
\usepackage{cutwin}
\usepackage{enumerate}
\usepackage{enumitem}
\usepackage{epstopdf}
\usepackage{graphicx}
\usepackage{ifthen}
\usepackage{lipsum}
\usepackage{mathrsfs}	
\usepackage{multimedia}
\usepackage{wrapfig}
}
\bibliographystyle{humanbio}


\usepackage[utf8]{inputenc}

\newcommand{\itemlist}[1]{\begin{itemize}#1\end{itemize}}
\newcommand{\enumlist}[1]{\begin{enumerate}#1\end{enumerate}}
\newcommand{\desclist}[1]{\begin{description}#1\end{description}}
\newcommand\tab[1][0.5cm]{\hspace*{#1}}

\newcommand{\Answer}[1]{\begin{quote}{\color{blue}#1}\end{quote}}
\newcommand{\AND}{\wedge}
\newcommand{\OR}{\vee}
\newcommand{\ra}{\rightarrow}
\newcommand{\lra}{\leftrightarrow}

\title {{\bf ECE 471 Lab 3} \\
\large{MD5 Collision Attack Lab}}

\author{Mitchell Dzurick}
\date{3/9/2020}
\begin{document}

\maketitle
\textbf{Github with all documentation - \url{https://www.github.com/mitchdz/ECE471}}
\tableofcontents 

\clearpage


Secret Key Encryption Lab

Copyright © 2018 Wenliang Du, Syracuse University. The development of this document was partially funded by the National
Science Foundation under Award No. 1303306 and 1718086. This work is licensed under a Creative Commons
Attribution-NonCommercial- ShareAlike 4.0 International License. A human-readable summary of (and not a substitute for)
the license is the following: You are free to copy and redistribute the material in any medium or format. You must give
appropriate credit. If you remix, transform, or build upon the material, you must distribute your contributions under the
same license as the original. You may not use the material for commercial purposes.

\section{Overview}

A secure one-way hash function needs to satisfy two properties: the one-way property and the collision-
resistance property. The one-way property ensures that given a hash value h, it is computationally infeasible
to find an input M , such that hash(M ) = h. The collision-resistance property ensures that it is compu-
tationally infeasible to find two different inputs M 1 and M 2 , such that hash(M 1 ) = hash(M 2 ).
Several widely-used one-way hash functions have trouble maintaining the collision-resistance prop-
erty. At the rump session of CRYPTO 2004, Xiaoyun Wang and co-authors demonstrated a collision attack
against MD5 [1]. In February 2017, CWI Amsterdam and Google Research announced the SHAttered at-
tack, which breaks the collision-resistance property of SHA-1 [3]. While many students do not have trouble
understanding the importance of the one-way property, they cannot easily grasp why the collision-resistance
property is necessary, and what impact these attacks can cause.
The learning objective of this lab is for students to really understand the impact of collision attacks, and
see in first hand what damages can be caused if a widely-used one-way hash function’s collision-resistance
property is broken. To achieve this goal, students need to launch actual collision attacks against the MD5
hash function. Using the attacks, students should be able to create two different programs that share the
same MD5 hash but have completely different behaviors. This lab covers a number of topics described in
the following:


    \begin{itemize}
        \item One-way hash function
        \item The collision-resistance property
        \item Collision attacks
        \item MD5
    \end{itemize}

\textbf{Lab Environment}. This lab has been tested on our pre-built Ubuntu 12.04 VM and Ubuntu 16.04 VM, both of which
can be downloaded from the SEED website.

\clearpage
\section{Lab Tasks}
\section{Task 1: Generating Two Different Files with the Same MD5 Hash}
In this task, we will generate two different files with the same MD5 hash values. The beginning parts of these
two files need to be the same, i.e., they share the same prefix. We can achieve this using the md5collgen
program, which allows us to provide a prefix file with any arbitrary content. The way how the program works
is illustrated in Figure 1. The following command generates two output files, out1.bin and out2.bin,
for a given a prefix file \emph{prefix.txt}:

\begin{verbatim}
$ md5collgen -p prefix.txt -o out1.bin out2.bin
\end{verbatim}

\begin{figure}[H]
	\begin{center}
		\includegraphics[scale=0.6]{pics/i1.png}
	\end{center}{}
	\caption{Running the key generation program without srand multiple times}
	\label{fig:i1}
\end{figure}

We can check whether the output files are distinct or not using the diff command. We can also use the
md5sum command to check the MD5 hash of each output file. See the following commands.
\begin{verbatim}
$ diff out1.bin out2.bin
$ md5sum out1.bin
$ md5sum out2.bin
\end{verbatim}

\tab Since out1.bin and out2.bin are binary, we cannot view them using a text-viewer program, such
as cat or more; we need to use a binary editor to view (and edit) them. We have already installed a hex
editor software called bless in our VM. Please use such an editor to view these two output files, and
describe your observations. In addition, you should answer the following questions:
\begin{itemize}
	\item Question 1. If the length of your prefix file is not multiple of 64, what is going to happen?
	\item Question 2. Create a prefix file with exactly 64 bytes, and run the collision tool again, and see what happens.
	\item Question 3. Are the data (128 bytes) generated by md5collgen completely different for the two
output files? Please identify all the bytes that are different.	
\end{itemize}



\subsection{Task 1 Solution}
\subsubsection{Task 1 Question 1 Solution}
the prefix was padded with zero bytes until the size is a multiple of 64.
\subsubsection{Task 1 Question 2 Solution}
With 64-byte prefix, no bytes are needed for padding. The prefix then has extra data added on for the collision.
\subsubsection{Task 1 Question 3 Solution}

The md5collgen program is used for the first time in this task, so some initial observations are made.

\begin{figure}[H]
	\begin{center}
		\includegraphics[scale=0.65]{pics/t1p0.png}
	\end{center}{}
	\caption{Running the key generation program without srand multiple times}
	\label{fig:t1p0}
\end{figure}


Figure~\ref{fig:t1p0} shows the results of md5collgen being ran on a file called prefix.txt that has the contents `Hello World!` and the prefix is seen to be padded to 64 bytes with bytes of 0 bits.


\begin{figure}[H]
	\begin{center}
		\includegraphics[scale=0.65]{pics/t1p1.png}
	\end{center}{}
	\caption{64 bytes of prefix being put into the md5collgen}
	\label{fig:t1p1}
\end{figure}

Figure~\ref{fig:t1p1} shows that the files don't completely differ, but they do differ in certain areas. It's actually apparent that only 8 bytes are different out of the 192 bytes that are output. The bytes that changed only changed a little bit as well. They only differ in one bit.


\subsection{Task 2: Understanding MD5's Property}
\subsubsection{Task 2: Solution}




\end{document}

