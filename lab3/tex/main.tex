\documentclass[12pt]{article}
\usepackage[margin=1in]{geometry}                % See geometry.pdf to learn the layout options. There are lots.
\geometry{letterpaper}                   % ... or a4paper or a5paper or ... 
%\geometry{landscape}                % Activate for for rotated page geometry
\usepackage[parfill]{parskip}    % Activate to begin paragraphs with an empty line rather than an indent

%%%%%%%%%%%%%%%%%%%%
\newcommand{\hide}[1]{}



\usepackage{natbib}
\usepackage{xcolor}
\usepackage{url}
\usepackage{hyperref}
\usepackage{mathtools}
\usepackage[utf8]{inputenc}
\usepackage{float}


\hide{
\usepackage{amscd}
\usepackage{amsfonts}
\usepackage{amsmath}
\usepackage{amssymb}
\usepackage{amsthm}
\usepackage{cases}		 
\usepackage{cutwin}
\usepackage{enumerate}
\usepackage{enumitem}
\usepackage{epstopdf}
\usepackage{graphicx}
\usepackage{ifthen}
\usepackage{lipsum}
\usepackage{mathrsfs}	
\usepackage{multimedia}
\usepackage{wrapfig}
}
\bibliographystyle{humanbio}


\usepackage[utf8]{inputenc}

\newcommand{\itemlist}[1]{\begin{itemize}#1\end{itemize}}
\newcommand{\enumlist}[1]{\begin{enumerate}#1\end{enumerate}}
\newcommand{\desclist}[1]{\begin{description}#1\end{description}}
\newcommand\tab[1][0.5cm]{\hspace*{#1}}

\newcommand{\Answer}[1]{\begin{quote}{\color{blue}#1}\end{quote}}
\newcommand{\AND}{\wedge}
\newcommand{\OR}{\vee}
\newcommand{\ra}{\rightarrow}
\newcommand{\lra}{\leftrightarrow}

\title {{\bf ECE 471 Lab 3} \\
\large{MD5 Collision Attack Lab}}

\author{Mitchell Dzurick}
\date{3/9/2020}
\begin{document}

\maketitle
\textbf{Github with all documentation - \url{https://www.github.com/mitchdz/ECE471}}
\tableofcontents 

\clearpage


Secret Key Encryption Lab

Copyright © 2018 Wenliang Du, Syracuse University. The development of this document was partially funded by the National
Science Foundation under Award No. 1303306 and 1718086. This work is licensed under a Creative Commons
Attribution-NonCommercial- ShareAlike 4.0 International License. A human-readable summary of (and not a substitute for)
the license is the following: You are free to copy and redistribute the material in any medium or format. You must give
appropriate credit. If you remix, transform, or build upon the material, you must distribute your contributions under the
same license as the original. You may not use the material for commercial purposes.

\section{Overview}

Generating random numbers is a quite common task in security software. In many cases, encryption keys
are not provided by users, but are instead generated inside the software. Their randomness is extremely
important; otherwise, attackers can predict the encryption key, and thus defeat the purpose of encryption.
Many developers know how to generate random numbers (e.g. for Monte Carlo simulation) from their
prior experiences, so they use the similar methods to generate the random numbers for security purpose.
Unfortunately, a sequence of random numbers may be good for Monte Carlo simulation, but they may be
bad for encryption keys. Developers need to know how to generate secure random numbers, or they will
make mistakes. Similar mistakes have been made in some well-known products, including Netscape and
Kerberos. \\
\tab In this lab, students will learn why the typical random number generation method is not appropriate
for generating secrets, such as encryption keys. They will further learn a standard way to generate pseudo
random numbers that are good for security purposes. This lab covers the following topics:


    \begin{itemize}
        \item Pseudo random number generation
        \item Mistakes in random number generation
        \item Generating encryption key
        \item The /dev/random and /dev/urandom device files
    \end{itemize}

\textbf{Lab Environment}. This lab has been tested on our pre-built Ubuntu 12.04 VM and Ubuntu 16.04 VM, both of which
can be downloaded from the SEED website.

\clearpage

\section{Lab Tasks}


\end{document}

