\documentclass[12pt]{article}
\usepackage[margin=1in]{geometry}                % See geometry.pdf to learn the layout options. There are lots.
\geometry{letterpaper}                   % ... or a4paper or a5paper or ... 
%\geometry{landscape}                % Activate for for rotated page geometry
\usepackage[parfill]{parskip}    % Activate to begin paragraphs with an empty line rather than an indent

%%%%%%%%%%%%%%%%%%%%
\newcommand{\hide}[1]{}



\usepackage{natbib}
\usepackage{xcolor}
\usepackage{url}
\usepackage{hyperref}
\usepackage{mathtools}
\usepackage[utf8]{inputenc}
\usepackage{float}
\usepackage{listings}
\usepackage{xcolor}
\usepackage{seqsplit}


\hide{
\usepackage{amscd}
\usepackage{amsfonts}
\usepackage{amsmath}
\usepackage{amssymb}
\usepackage{amsthm}
\usepackage{cases}		 
\usepackage{cutwin}
\usepackage{enumerate}
\usepackage{enumitem}
\usepackage{epstopdf}
\usepackage{graphicx}
\usepackage{ifthen}
\usepackage{lipsum}
\usepackage{mathrsfs}	
\usepackage{multimedia}
\usepackage{wrapfig}
}
\bibliographystyle{humanbio}


\usepackage[utf8]{inputenc}

\newcommand{\itemlist}[1]{\begin{itemize}#1\end{itemize}}
\newcommand{\enumlist}[1]{\begin{enumerate}#1\end{enumerate}}
\newcommand{\desclist}[1]{\begin{description}#1\end{description}}
\newcommand\tab[1][0.5cm]{\hspace*{#1}}

\newcommand{\Answer}[1]{\begin{quote}{\color{blue}#1}\end{quote}}
\newcommand{\AND}{\wedge}
\newcommand{\OR}{\vee}
\newcommand{\ra}{\rightarrow}
\newcommand{\lra}{\leftrightarrow}

\title {{\bf ECE 471 Lab 4} \\
\large{RSA Public-Key Encryption and Signature Lab}}

\author{Mitchell Dzurick}
\date{3/23/2020}
\begin{document}

\maketitle
\textbf{Github with all documentation - \url{https://www.github.com/mitchdz/ECE471}}
\tableofcontents 

\clearpage


RSA Public-Key Encryption and Signature Lab

Copyright © 2018 Wenliang Du, Syracuse University. The development of this document was partially funded by the National
Science Foundation under Award No. 1303306 and 1718086. This work is licensed under a Creative Commons
Attribution-NonCommercial- ShareAlike 4.0 International License. A human-readable summary of (and not a substitute for)
the license is the following: You are free to copy and redistribute the material in any medium or format. You must give
appropriate credit. If you remix, transform, or build upon the material, you must distribute your contributions under the
same license as the original. You may not use the material for commercial purposes.

\section{Overview}

RSA (RivestShamirAdleman) is one of the first public-key cryptosystems and is widely used for secure
communication. The RSA algorithm first generates two large random prime numbers, and then use them
to generate public and private key pairs, which can be used to do encryption, decryption, digital signature
generation, and digital signature verification. The RSA algorithm is built upon number theories, and it can
\\
\tab be quite easily implemented with the support of libraries.
The learning objective of this lab is for students to gain hands-on experiences on the RSA algorithm.
From lectures, students should have learned the theoretic part of the RSA algorithm, so they know mathematically how to generate public/private keys and how to perform encryption/decryption and signature
generation/verification. This lab enhances student’s understanding of RSA by requiring them to go through
every essential step of the RSA algorithm on actual numbers, so they can apply the theories learned from
the class. Essentially, students will be implementing the RSA algorithm using the C program language. The
lab covers the following security-related topics:

    \begin{itemize}
        \item Public-key cryptography
        \item The RSA algorithm and key generation
        \item Big number calculation
        \item Encryption and Decryption using RSA
        \item Digital signature
        \item X.509 certificate
    \end{itemize}

\textbf{Lab Environment}. This lab has been tested on our pre-built Ubuntu 12.04 VM and Ubuntu 16.04 VM, both of which
can be downloaded from the SEED website.

\clearpage
\section{lab flow}
This entire lab is done with a single program titled $flow.c$. The program can be found here  \url{https://github.com/mitchdz/ECE471/tree/master/lab4} with instructions on how to run. flow.c also utilizes a file titled l4$\_$util.h which is a file with a few helper functions designed to make the flow more simple.

The code is pasted blow:
\begin{verbatim}
/* flow.c */
#include <stdio.h>
#include <string.h>
#include <openssl/bn.h>
#include "l4_util.h"

int main ()
{
printf("Lab 4 RSA Public-Key Encryption and Signature Lab\n");
// part 3.1 Deriving a private key
BIGNUM *p = BN_new();
BIGNUM *q = BN_new();
BIGNUM *e = BN_new();

// Initialize p, q, e
BN_dec2bn(&p, "F7E75FDC469067FFDC4E847C51F452DF");
BN_dec2bn(&q, "E85CED54AF57E53E092113E62F436F4F");
BN_dec2bn(&e, "0D88C3"); //modulus

BIGNUM *part1PrivatKey = RSA_get_priv(p, q, e);
printf("(3.1)\n");
printBN("Task 1 private key ", part1PrivatKey);
printf("\n");

// Part 3.2
printf("(3.2)\n");
BIGNUM* enc = BN_new();
BIGNUM* dec = BN_new();
// Init private key d
BIGNUM* privateKey3_2 = BN_new();
BN_hex2bn(&privateKey3_2, DHEX3_2);

BIGNUM* publicKey = BN_new();
BN_hex2bn(&publicKey, NHEX3_2);
printBN("Public key: ", publicKey);
printf("\n");

BIGNUM* mod = BN_new();
BN_hex2bn(&mod, EHEX3_2);

BIGNUM* message = BN_new();
BN_hex2bn(&message, MHEX3_2);

printBN("Message Hex:", message);
enc = RSA_ENC(message, mod, publicKey);
printBN("Encrypted message:", enc);
dec = RSA_DEC(enc, privateKey3_2, publicKey);
printf("Decrypted message: ");
printHEX(BN_bn2hex(dec));
printf("\n");

/* Part 3.3 */
printf("(3.3)\n");
BIGNUM* task3_C = BN_new();
BN_hex2bn(&task3_C, CHEX3_3);

dec = RSA_DEC(task3_C, privateKey3_2, publicKey);
printf("Decrypted message: "); printHEX(BN_bn2hex(dec)); printf("\n");

/* part 3.4 Signing a Message */
printf("(3.4)\n");
BIGNUM* BN_t4 = BN_new();

// python -c ’print("I owe you $2000".encode("hex"))’
BN_hex2bn(&BN_t4, "49206f776520796f75202432303030"); 
enc = RSA_ENC(BN_t4, privateKey3_2, publicKey);
printBN("Signature: ", enc);
dec = RSA_DEC(enc, mod, publicKey);
printf("message: "); printHEX(BN_bn2hex(dec));

// python -c ’print("I owe you $3000".encode("hex"))’
BN_hex2bn(&BN_t4, "49206f776520796f75202433303030");
enc = RSA_ENC(BN_t4, privateKey3_2, publicKey);
printBN("Signature: ", enc);
dec = RSA_DEC(enc, mod, publicKey);
printf("message: "); printHEX(BN_bn2hex(dec)); printf("\n");


/* Part 3.5 Verifying a signature */
printf("(3.5)\n");
BIGNUM *S = BN_new();

BN_hex2bn(&publicKey, NHEX3_5);
BN_hex2bn(&S, SHEX3_5);

// correct signature
dec = RSA_DEC(S, mod, publicKey);
printf("message (regular) : "); printHEX(BN_bn2hex(dec)); printf("\n");

// corrupted signature
BN_hex2bn(&S, SHEX3_5p2);
dec = RSA_DEC(S, mod, publicKey);
printHEX(BN_bn2hex(dec)); printf("\n");

/* Part 3.6 Manually Verifying an X.509 Certificate */
//github.com cert
printf("(3.6)\n");

// Assign the public key
BIGNUM* t6_pub_key = BN_new();
BN_hex2bn(&t6_pub_key, N6);
printBN("the public key is: ", t6_pub_key);

// mod
BIGNUM* t6_mod = BN_new();
BN_hex2bn(&t6_mod, E6);

// signature
BIGNUM* BN_t6 = BN_new();
BN_hex2bn(&BN_t6, S6);

// This value should match the pre-comptued hash
BIGNUM* t6_dec = RSA_DEC(BN_t6, t6_mod, t6_pub_key);
printBN("the hash for task6 is: ", t6_dec);
printf("\n");
printf("pre-computed hash: ");
printf(calcHASH6);
printf("\n");

return 0;
}

\end{verbatim}

\begin{verbatim}
//4_util.h
/* l4_util.h */
#include <stdio.h>
#include <string.h>
#include <string.h>

#define PHEX3_1 "F7E75FDC469067FFDC4E847C51F452DF"
#define QHEX3_1 "E85CED54AF57E53E092113E62F436F4F"
#define EHEX3_1 "0D88C3"

#define DHEX3_2 "74D806F9F3A62BAE331FFE3F0A68AFE35B3D2E4794148AACBC26AA381CD7D30D"
#define NHEX3_2 "DCBFFE3E51F62E09CE7032E2677A78946A849DC4CDDE3A4D0CB81629242FB1A5"
#define MHEX3_2 "4120746f702073656372657421"
#define EHEX3_2 "010001"

#define CHEX3_3 "8C0F971DF2F3672B28811407E2DABBE1DA0FEBBBDFC7DCB67396567EA1E2493F"

#define SIGHEX3_4 "239a09ea0d5fdaea94ec97130b1c74c89764226065bbe614da7fb9f851be7beabd5f8ece4b06c84c8bf49162e2c914fc8c8b6bd98b9ed086fb3bdb7f74fc5075424541fc8911c6c87953d5377d38fff94ecb0cb86018d3f00c5a1d418cf9e65e19dd56a947ae301fc73ffc9d513ece60e85373a4408f6958a466e41044a3b924d3e85c4ea6afaa86b2cfe388f70873d8513c1c10df65ccdfdb00cbd7d7a5f08dc7f3e5f40b500ac7913796dd11a71d88aaf53f728ab1584b80ea7e32e378679b247297df7b562b08efd4a56e1861f41dcd1d17d58e1015a8c5c580862167e884876ce72ff539e8da7397bd86cca42ac02576b778ad4a96943c6ae1b33f27a8dd"
#define MHEX3_4p1 "I owe you $2000"
#define MHEX3_4p2 "I owe you $3000"

#define MHEX3_5 "Launch a missle."
#define SHEX3_5 "643D6F34902D9C7EC90CB0B2BCA36C47FA37165C0005CAB026C0542CBDB6802F"
#define SHEX3_5p2 "643D6F34902D9C7EC90CB0B2BCA36C47FA37165C0005CAB026C0542CBDB6803F"
#define EHEX3_5 "010001"
#define NHEX3_5 "AE1CD4DC432798D933779FBD46C6E1247F0CF1233595113AA51B450F18116115"

#define N6 "D753A40451F899A616484B6727AA9349D039ED0CB0B00087F1672886858C8E63DABCB14038E2D3F5ECA50518B83D3EC5991732EC188CFAF10CA6642185CB071034B052882B1F689BD2B18F12B0B3D2E7881F1FEF387754535F80793F2E1AAAA81E4B2B0DABB763B935B77D14BC594BDF514AD2A1E20CE29082876AAEEAD764D69855E8FDAF1A506C54BC11F2FD4AF29DBB7F0EF4D5BE8E16891255D8C07134EEF6DC2DECC48725868DD821E4B04D0C89DC392617DDF6D79485D80421709D6F6FFF5CBA19E145CB5657287E1C0D4157AAB7B827BBB1E4FA2AEF2123751AAD2D9B86358C9C77B573ADD8942DE4F30C9DEEC14E627E17C0719E2CDEF1F910281933"
#define E6 "010001"
#define S6 "700f5a96a758e5bf8a9da827982b007f26a907daba7b82544faf69cfbcf259032bf2d5745825d81ea42076626029732ad7dccc6f77856bca6d24f83513473fd2e2690a9d342d7b7b9bcd1e75d5506c3ecb1ca330b1aa9207a93a767645bd7891c4ce1a9e22e40b89bae68cc17982a3b8d4c0fc1f2ded4d5255412aa83a2cad0772ae0ad2c667c44f07171899f765a95760155a344c11cff6cf6b213680efc6f15463263539eebbc483649b240a73eca0481673c8b9d7485556987af7bb975c69a406180478dafe9876be222f7f0777874e88199af855ec5c122a5948db493e155e675aa25eeecc53288c0e33931403640bc5e5780994015a75fc929dafed7a29"
#define calcHASH6 "85088f934d3e58e3673ea5be32c7c8cf6965e4ab93fbed4fff634723f46d5693"




#define NBITS 256

BIGNUM *RSA_get_priv(BIGNUM* p, BIGNUM* q, BIGNUM* e){
BN_CTX *ctx = BN_CTX_new();
BIGNUM* p_minus_one = BN_new();
BIGNUM* q_minus_one = BN_new();
BIGNUM* one = BN_new();
BIGNUM* ret = BN_new();

BN_dec2bn(&one, "1");
BN_sub(p_minus_one, p, one);
BN_sub(q_minus_one, q, one);
BN_mul(ret, p_minus_one, q_minus_one, ctx);

BIGNUM* res = BN_new();
BN_mod_inverse(res, e, ret, ctx);
BN_CTX_free(ctx);
return res;
}

void printBN(char *msg, BIGNUM * a)
{
/* Use BN_bn2hex(a) for hex string
*
* * Use BN_bn2dec(a) for decimal string */
char * number_str = BN_bn2hex(a);
printf("%s %s\n", msg, number_str);
OPENSSL_free(number_str);
}

BIGNUM* RSA_ENC(BIGNUM* message, BIGNUM* mod, BIGNUM* pub_key){
BN_CTX *ctx = BN_CTX_new();
BIGNUM* enc = BN_new();
BN_mod_exp(enc, message, mod, pub_key, ctx);
BN_CTX_free(ctx);
return enc;
}

BIGNUM* RSA_DEC(BIGNUM* enc, BIGNUM* priv_key, BIGNUM* pub_key){
BN_CTX *ctx = BN_CTX_new();
BIGNUM* dec = BN_new();
BN_mod_exp(dec, enc, priv_key, pub_key, ctx);
BN_CTX_free(ctx);
return dec;
}
int hex2int(char c){
// return (int)strtol(c, NULL, 16);
if (c >= 97)
c = c - 32;
int first = c / 16 - 3;
int second = c % 16;
int res = first * 10 + second;
if (res > 9) res--;
return res;
}

int hex2ascii(const char c, const char d){return (hex2int(c) * 16) + hex2int(d);}

void printHEX(const char* st){
int length = strlen(st);
if (length % 2 != 0) {
printf("%s\n", "hex length needs to be even.");
return;
}
int i;
char buf = 0;
for(i = 0; i < length; i++) {
if(i % 2 != 0)
printf("%c", hex2ascii(buf, st[i]));
else
buf = st[i];
}
printf("\n");
}

\end{verbatim}

The commands used in the makefile in the git directory to compile are as follows:
\begin{verbatim}
$ gcc src/flow.c -o bin/flow -lcrypto && ./bin/flow
\end{verbatim}
The output of running bin/flow are below:

\begin{figure}[H]
\begin{verbatim}
Lab 4 RSA Public-Key Encryption and Signature Lab
(3.1)
Task 1 private key  0x3587A24598E5F2A21DB007D89D18CC50ABA5075BA19A33890FE7C28A9B496AEB

(3.2)
Public key:  DCBFFE3E51F62E09CE7032E2677A78946A849DC4CDDE3A4D0CB81629242FB1A5

Message Hex: 4120746F702073656372657421
Encrypted message: 6FB078DA550B2650832661E14F4F8D2CFAEF475A0DF3A75CACDC5DE5CFC5FADC
Decrypted message: A top secret!

(3.3)
Decrypted message: Password is dees

(3.4)
Signature:  80A55421D72345AC199836F60D51DC9594E2BDB4AE20C804823FB71660DE7B82
message: I owe you $2000
Signature:  04FC9C53ED7BBE4ED4BE2C24B0BDF7184B96290B4ED4E3959F58E94B1ECEA2EB
message: I owe you $3000

(3.5)
message (regular) : Launch a missile.

message (corruped) : <corrupted output>

\end{verbatim}
the public key is:  \seqsplit{753A40451F899A616484B6727AA9349D039ED0CB0B00087F1672886858C8E63DABCB14038E2D3F5ECA50518B83D3EC5991732EC188CFAF10CA6642185CB071034B052882B1F689BD2B18F12B0B3D2E7881F1FEF387754535F80793F2E1AAAA81E4B2B0DABB763B935B77D14BC594BDF514AD2A1E20CE29082876AAEEAD764D69855E8FDAF1A506C54BC11F2FD4AF29DBB7F0EF4D5BE8E16891255D8C07134EEF6DC2DECC48725868DD821E4B04D0C89DC392617DDF6D79485D80421709D6F6FFF5CBA19E145CB5657287E1C0D4157AAB7B827BBB1E4FA2AEF2123751AAD2D9B86358C9C77B573ADD8942DE4F30C9DEEC14E627E17C0719E2CDEF1F910281933} \\
the hash for task6 is:  \seqsplit{01FFFFFFFFFFFFFFFFFFFFFFFFFFFFFFFFFFFFFFFFFFFFFFFFFFFFFFFFFFFFFFFFFFFFFFFFFFFFFFFFFFFFFFFFFFFFFFFFFFFFFFFFFFFFFFFFFFFFFFFFFFFFFFFFFFFFFFFFFFFFFFFFFFFFFFFFFFFFFFFFFFFFFFFFFFFFFFFFFFFFFFFFFFFFFFFFFFFFFFFFFFFFFFFFFFFFFFFFFFFFFFFFFFFFFFFFFFFFFFFFFFFFFFFFFFFFFFFFFFFFFFFFFFFFFFFFFFFFFFFFFFFFFFFFFFFFFFFFFFFFFFFFFFFFFFFFFFFFFFFFFFFFFFFFFFFFFFFFFFFFFFFFFFFFFFFFFFFFFFFFFFFFFFFFFFFFFFFFFFFFFFFFFFFFFFFFFFFFFFFFFFFF003031300D060960864801650304020105000420}{\seqsplit{85088F934D3E58E3673EA5BE32C7C8CF6965E4AB93FBED4FFF634723F46D5693}
	
pre-computed hash: 85088f934d3e58e3673ea5be32c7c8cf6965e4ab93fbed4fff634723f46d5693}
	\caption{output of running the flow program}
	\label{fig:flow}
\end{figure}


\section{Tasks}
\subsection{Task 1: Deriving the Private Key}
\subsubsection{Task 1: Solution}
The derived private key can be seen in Figure~\ref{fig:flow} where the output indicates the key from this section. The key is derived through a custom function named BIGNUM *RSA$\_$get$\_$priv(BIGNUM* p, BIGNUM* q, BIGNUM* e). The function is shown below.

\begin{verbatim}
BIGNUM *RSA_get_priv(BIGNUM* p, BIGNUM* q, BIGNUM* e){
  BN_CTX *ctx = BN_CTX_new();
  BIGNUM* p_minus_one = BN_new();
  BIGNUM* q_minus_one = BN_new();
  BIGNUM* one = BN_new();
  BIGNUM* ret = BN_new();

  BN_dec2bn(&one, "1");  
  BN_sub(p_minus_one, p, one);
  BN_sub(q_minus_one, q, one);
  // (p - 1)(q - 1) also called the totient
  BN_mul(ret, p_minus_one, q_minus_one, ctx);

  BIGNUM* res = BN_new();
  BN_mod_inverse(res, e, ret, ctx);
  BN_CTX_free(ctx);
  return res;
}
\end{verbatim}

The function above calculates $\Phi$(N) = (p - 1)(q - 1) and then uses the totient to calculate d with a reverse modulus operation.


\subsection{Task 2: Encrypting a Message}
\subsubsection{Task 2: Solution}
The output of flow.c shown in Figure~\ref{fig:flow} is as follows:
\begin{verbatim}
(3.2)
Public key:  DCBFFE3E51F62E09CE7032E2677A78946A849DC4CDDE3A4D0CB81629242FB1A5

Message Hex: 4120746F702073656372657421
Encrypted message: 6FB078DA550B2650832661E14F4F8D2CFAEF475A0DF3A75CACDC5DE5CFC5FADC
Decrypted message: A top secret!
\end{verbatim}

Where the message "A top secret!" is encrypted to "6FB078DA550B2650832661E14F4F8D2CFAEF475A0DF3A75CACDC5DE5CFC5FADC".

The code that encrypts is shown below:
\begin{verbatim}
BIGNUM* RSA_ENC(BIGNUM* message, BIGNUM* mod, BIGNUM* pub_key){
BN_CTX *ctx = BN_CTX_new();
BIGNUM* enc = BN_new();
BN_mod_exp(enc, message, mod, pub_key, ctx);
BN_CTX_free(ctx);
return enc;
}
\end{verbatim}
Which is a very simple operation that follows the following algorithm:

\begin{center} $enc = message^{mod} \% pub\_key$ \end{center}

Here is the following function given by \url{openssl.org/docs/man1.1.0/man3/BN_mod_exp.html}:
\begin{verbatim}
 int BN_mod_exp(BIGNUM *r, BIGNUM *a, const BIGNUM *p,
\end{verbatim}

where BN$\_$mod$\_$exp() computes a to the p-th power modulo m (r=$a^p$ $\%$ m).

\subsection{Task 3: Decrypting a Message}
\subsubsection{Task 3: Solution}
The decrypted message is shown in Figure~\ref{fig:flow} as:

Decrypted message: Password is dees

This is found through the custom function created below:
\begin{verbatim}
BIGNUM* RSA_DEC(BIGNUM* enc, BIGNUM* priv_key, BIGNUM* pub_key){
  BN_CTX *ctx = BN_CTX_new();
  BIGNUM* dec = BN_new();
  BN_mod_exp(dec, enc, priv_key, pub_key, ctx);
  BN_CTX_free(ctx);
  return dec;
}
\end{verbatim}
This function is extremely similar to Task 2's solution in that we use BN$\_$mod$\_$exp. The only difference now is that we find the message through the following algorithm:

\begin{center} $message = enc^{priv\_key} \% pub\_key$ \end{center}

which is just the inverse of encryption.
\subsection{Task 4: Signing a Message}
\subsubsection{Task 4: Solution}
The output for Task 4 is shown below:
\begin{verbatim}
Signature:  80A55421D72345AC199836F60D51DC9594E2BDB4AE20C804823FB71660DE7B82
message: I owe you $2000
Signature:  04FC9C53ED7BBE4ED4BE2C24B0BDF7184B96290B4ED4E3959F58E94B1ECEA2EB
message: I owe you $3000
\end{verbatim}

Where the signature is wildly different for the very similar message.

The code segment for signing a message is shown below.
\begin{verbatim}
/* part 3.4 Signing a Message */
printf("(3.4)\n");
BIGNUM* BN_t4 = BN_new();

// python -c ’print("I owe you $2000".encode("hex"))’
BN_hex2bn(&BN_t4, "49206f776520796f75202432303030"); 
enc = RSA_ENC(BN_t4, privateKey3_2, publicKey);
printBN("Signature: ", enc);
dec = RSA_DEC(enc, mod, publicKey);
printf("message: "); printHEX(BN_bn2hex(dec));

// python -c ’print("I owe you $3000".encode("hex"))’
BN_hex2bn(&BN_t4, "49206f776520796f75202433303030");
enc = RSA_ENC(BN_t4, privateKey3_2, publicKey);
printBN("Signature: ", enc);
dec = RSA_DEC(enc, mod, publicKey);
printf("message: "); printHEX(BN_bn2hex(dec)); printf("\n");
\end{verbatim}

To sign the message is pretty straight forward. We get the message in BN$\_$t4 and then encrypt that message with the private and public key. The same is done for a slightly modified message.

\subsection{Task 5: Verifying a Signature}
\subsubsection{Task 5: Solution}
The output for Task 5 is shown:
\begin{verbatim}
message (regular) : Launch a missile.
message (corruped) : <corrupted output>
\end{verbatim}

The message is verified because it shows the message "Launch a missile." The slightly modified signature (by even one bit) shows very corrupted output. The code segment for this section is shown below.

\begin{verbatim}
printf("(3.5)\n");
BIGNUM *S = BN_new();

BN_hex2bn(&publicKey, NHEX3_5);
BN_hex2bn(&S, SHEX3_5);

// correct signature
dec = RSA_DEC(S, mod, publicKey);
printf("message (regular) : "); printHEX(BN_bn2hex(dec)); printf("\n");

// corrupted signature
BN_hex2bn(&S, SHEX3_5p2);
dec = RSA_DEC(S, mod, publicKey);
printHEX(BN_bn2hex(dec)); printf("\n");
\end{verbatim}

where NHEX3$\_$35 and NHEX3$\_$35 reside in the helper file as defines. The code is very straightforward and similar to Task 4.

\subsection{Task 6: Manually Verifying an X.509 Certificate}
\subsubsection{Task 6: Solution}


Figure~\ref{fig:flow} shows the result of task 3.6 which is copied below:


the public key is:  \seqsplit{753A40451F899A616484B6727AA9349D039ED0CB0B00087F1672886858C8E63DABCB14038E2D3F5ECA50518B83D3EC5991732EC188CFAF10CA6642185CB071034B052882B1F689BD2B18F12B0B3D2E7881F1FEF387754535F80793F2E1AAAA81E4B2B0DABB763B935B77D14BC594BDF514AD2A1E20CE29082876AAEEAD764D69855E8FDAF1A506C54BC11F2FD4AF29DBB7F0EF4D5BE8E16891255D8C07134EEF6DC2DECC48725868DD821E4B04D0C89DC392617DDF6D79485D80421709D6F6FFF5CBA19E145CB5657287E1C0D4157AAB7B827BBB1E4FA2AEF2123751AAD2D9B86358C9C77B573ADD8942DE4F30C9DEEC14E627E17C0719E2CDEF1F910281933} \\
the hash for task6 is:  \seqsplit{01FFFFFFFFFFFFFFFFFFFFFFFFFFFFFFFFFFFFFFFFFFFFFFFFFFFFFFFFFFFFFFFFFFFFFFFFFFFFFFFFFFFFFFFFFFFFFFFFFFFFFFFFFFFFFFFFFFFFFFFFFFFFFFFFFFFFFFFFFFFFFFFFFFFFFFFFFFFFFFFFFFFFFFFFFFFFFFFFFFFFFFFFFFFFFFFFFFFFFFFFFFFFFFFFFFFFFFFFFFFFFFFFFFFFFFFFFFFFFFFFFFFFFFFFFFFFFFFFFFFFFFFFFFFFFFFFFFFFFFFFFFFFFFFFFFFFFFFFFFFFFFFFFFFFFFFFFFFFFFFFFFFFFFFFFFFFFFFFFFFFFFFFFFFFFFFFFFFFFFFFFFFFFFFFFFFFFFFFFFFFFFFFFFFFFFFFFFFFFFFFFFFF003031300D060960864801650304020105000420}\textbf{\seqsplit{85088F934D3E58E3673EA5BE32C7C8CF6965E4AB93FBED4FFF634723F46D5693}}

pre-computed hash: 85088f934d3e58e3673ea5be32c7c8cf6965e4ab93fbed4fff634723f46d5693

The website used for this part is \url{github.com}. It can easily be seen that the pre-computed hash occures in the calculated hash near the end. The other characters before the has is just noise. 

The steps to run this task are as follows:
\begin{verbatim}
openssl s_client -connect www.github.com:443 -showcerts > openssl_dump.txt
\end{verbatim}

The certificate for the CA authority and github are then copied into c0.pem and c1.pem. Github's cert is going to be the first listed cert whereas the CA authority will always be the second listed cert.

We then extract the modulus from the CA cert
\begin{verbatim}
openssl x509 -in c1.pem -noout -modulus
\end{verbatim}

The exponent is found in the long listing:
\begin{verbatim}
openssl x509 -in c1.pem -text -noout
\end{verbatim}

Manually extracting the signature from github with
\begin{verbatim}
openssl x509 -in c0.pem -text -noout
\end{verbatim}

And then removing all the spaces and colons with the bash command
\begin{verbatim}
cat signature | tr -d ’[:space:]:’
\end{verbatim}

Finally, we need to compute the hash so we can compare it to the program's value.

\begin{verbatim}
openssl asn1parse -i -in c0.pem -strparse 4 -out c0_body.bin -noout
sha256sum c0_body.bin
\end{verbatim}


Now we have the signature, the modulus, the exponent, and a pre-computed hash. The hash is computed with the following line of code in the program
\begin{verbatim}
BIGNUM* t6_dec = RSA_DEC(BN_t6, t6_mod, t6_pub_key);
\end{verbatim}

which does match the pre-computed hash value!


\end{document}

