\documentclass[12pt]{article}
\usepackage[margin=1in]{geometry}                % See geometry.pdf to learn the layout options. There are lots.
\geometry{letterpaper}                   % ... or a4paper or a5paper or ... 
%\geometry{landscape}                % Activate for for rotated page geometry
\usepackage[parfill]{parskip}    % Activate to begin paragraphs with an empty line rather than an indent

%%%%%%%%%%%%%%%%%%%%
\newcommand{\hide}[1]{}



\usepackage{natbib}
\usepackage{xcolor}
\usepackage{url}
\usepackage{hyperref}
\usepackage{mathtools}
\usepackage[utf8]{inputenc}
\usepackage{float}
\usepackage{listings}
\usepackage{xcolor}


\hide{
\usepackage{amscd}
\usepackage{amsfonts}
\usepackage{amsmath}
\usepackage{amssymb}
\usepackage{amsthm}
\usepackage{cases}		 
\usepackage{cutwin}
\usepackage{enumerate}
\usepackage{enumitem}
\usepackage{epstopdf}
\usepackage{graphicx}
\usepackage{ifthen}
\usepackage{lipsum}
\usepackage{mathrsfs}	
\usepackage{multimedia}
\usepackage{wrapfig}
}
\bibliographystyle{humanbio}


\usepackage[utf8]{inputenc}

\newcommand{\itemlist}[1]{\begin{itemize}#1\end{itemize}}
\newcommand{\enumlist}[1]{\begin{enumerate}#1\end{enumerate}}
\newcommand{\desclist}[1]{\begin{description}#1\end{description}}
\newcommand\tab[1][0.5cm]{\hspace*{#1}}

\newcommand{\Answer}[1]{\begin{quote}{\color{blue}#1}\end{quote}}
\newcommand{\AND}{\wedge}
\newcommand{\OR}{\vee}
\newcommand{\ra}{\rightarrow}
\newcommand{\lra}{\leftrightarrow}

\title {{\bf ECE 471 Lab 3} \\
\large{MD5 Collision Attack Lab}}

\author{Mitchell D6zurick}
\date{3/23/2020}
\begin{document}

\maketitle
\textbf{Github with all documentation - \url{https://www.github.com/mitchdz/ECE471}}
\tableofcontents 

\clearpage


RSA Public-Key Encryption and Signature Lab

Copyright © 2018 Wenliang Du, Syracuse University. The development of this document was partially funded by the National
Science Foundation under Award No. 1303306 and 1718086. This work is licensed under a Creative Commons
Attribution-NonCommercial- ShareAlike 4.0 International License. A human-readable summary of (and not a substitute for)
the license is the following: You are free to copy and redistribute the material in any medium or format. You must give
appropriate credit. If you remix, transform, or build upon the material, you must distribute your contributions under the
same license as the original. You may not use the material for commercial purposes.

\textbf{Overview}

RSA (RivestShamirAdleman) is one of the first public-key cryptosystems and is widely used for secure
communication. The RSA algorithm first generates two large random prime numbers, and then use them
to generate public and private key pairs, which can be used to do encryption, decryption, digital signature
generation, and digital signature verification. The RSA algorithm is built upon number theories, and it can
\\
\tab be quite easily implemented with the support of libraries.
The learning objective of this lab is for students to gain hands-on experiences on the RSA algorithm.
From lectures, students should have learned the theoretic part of the RSA algorithm, so they know mathematically how to generate public/private keys and how to perform encryption/decryption and signature
generation/verification. This lab enhances student’s understanding of RSA by requiring them to go through
every essential step of the RSA algorithm on actual numbers, so they can apply the theories learned from
the class. Essentially, students will be implementing the RSA algorithm using the C program language. The
lab covers the following security-related topics:

    \begin{itemize}
        \item Public-key cryptography
        \item The RSA algorithm and key generation
        \item Big number calculation
        \item Encryption and Decryption using RSA
        \item Digital signature
        \item X.509 certificate
    \end{itemize}

\textbf{Lab Environment}. This lab has been tested on our pre-built Ubuntu 12.04 VM and Ubuntu 16.04 VM, both of which
can be downloaded from the SEED website.

\clearpage
\section{Task 1: Deriving the Private Key}
\subsection{Task 1: Solution}


\clearpage
\section{Task 2: Encrypting a Message}
\subsection{Task 2: Solution}

\clearpage
\section{Task 3: Decrypting a Message}
\subsection{Task 3: Solution}

\clearpage
\section{Task 4: Signing a Message}
\subsection{Task 4: Solution}









\end{document}

